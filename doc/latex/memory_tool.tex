%文档类型
\documentclass[a4paper]{aritcle}

%引用包裹
\usepackage{bm}
\usepackage{cmap}
\usepackage{ctex}
\usepackage{cite}
\usepackage{color}
\usepackage{float}
\usepackage{xeCJK}
\usepackage{amsthm}
\usepackage{amsmath}
\usepackage{amssymb}
\usepackage{setspace}
\usepackage{geometry}
\usepackage{hyperref}
\usepackage{enumerate}
\usepackage{indentfirst}
\usepackage[cache=false]{minted}

%代码高亮
\geometry{margin=1in}

memory.hpp

\section{综述}
    一个便于从文件读写的“内存”管理方式 
    仿照stl中的smart_ptr,实现引用计数
    memory_pool是内存池,基于vector,pool_ptr是指向相应对象的指针

    所有成员都是静态成员
    用vector来实现一个内存池,int类型的数组下标充当指针。


\subsection{memory_pool}

\subsubsection{成员}
    
    container, recycler, counter, is_counted
    
    pool_ptr里面存储的是下标,

    vector<T> container
    存储实际内容

    vector<int> recycler
    存储已经没有指向它的指针的地址(下标),内存回收。

    vector<int> counter
    引用计数器,counter[i]的值(num)表示有(num)个pool_ptr指向container[i]这个对象


\subsubsection{成员函数}
    void save(QDataStream &out) 
    将该memory_pool中的内容输出到数据流out中,用于文件读写

    void load(QDataStream &in)
    从数据流中读入该memory_pool的内容,用于文件读写

\subsection{pool_ptr}

\subsubsection{成员变量}
    int pos
    指向的是container中的第pos个元素

\subsubsection{成员函数}
    void terminate()
    析构该指针,同时检查引用计数,如果已经是最后一个则销毁该对象(放入recycle)

    pool_ptr get_T() 
    构造一个指向T类型的指针,如果recycle中有元素则优先使用recycle中的
    void put_T()
    将一个指向T类型的指针放入recycle中

    文件读写pool_ptr的时候只读写pos(相当于一个int)

    T& operator *()
    返回相应对象的引用

    T* opeartor ->()
    返回该对象所在的内存地址