\section{Server}

\subsection{综述}
    server里面用map存储各种信息,同时实现了一些与相应的map交互的接口

    TTS(short for Train-Ticket-System)
    包装了server与前端交互的所有接口

\subsection{server}
    \subsubsection{成员变量}
    用map分别存储user, admin, line, city, Station
    \begin{itemize}
        \item map<int, user_ptr> users
        ID -> 指向user的指针
        \item map<int, admin_ptr> admins
        ID -> 指向admin的指针
        \item map<QString, line_ptr> lines
        车次号 -> 指向该line的指针
        \item map<QString, city_ptr> cities
        城市名 -> 指向该city的指针
        \item map<QString, station_ptr> stations
        车站名 -> 指向该station的指针
    \end{itemize}

    \subsubsection{接口}
    \begin{itemize}
    \item{find}
        返回相应指向相应类型对象的pool_ptr
        \begin{itemize}
        \item {find_city(const QString &name)}
        \item {find_user(const int &ID}
        \item {find_admin(const int &ID}
        \item {find_station(const QString &name}
        \item {find_line(const QString &name}
        \end{itemize}
    
    \item{check}
        返回某个city/user/...是否已经在map中存在
        \begin{itemize}
        \item {bool check_city(const QString& name)}
        传入一个城市名(字符串),检查这个城市是否已经存在于map中
        \item {bool check_user(const QString& name)}
        \item {bool check_admin(const QString& name)}
        \item {bool check_station(const QString& name)}
        \item {bool check_line(const QString& name)}
        \end{itemize}
    
    \item{add}
        将某个对象插入到相应的map中(指向该对象的指针)
        \begin{itemize}
        \item {bool add_city(const QString& name)}
        \item {bool add_user(const QString& name)}
        \item {bool add_admin(const QString& name)}
        \item {bool add_station(const QString& name)}
        \item {bool add_line(const QString& name)}
        \end{itemize}
    \item{delete}
        \begin{itemize}
        \item void delete_line(const QString & line)
        删除该线路,传入参数是该line的车次
        \end{itemize}
    \end{itemize}

\subsection{TTS}
    \subsubsection{成员变量}
    \begin{itemize}
    \item (Server)server
    server对象
    \item (int) id_cnt
    id的全局计数器
    \item (deque<Log>) logs
    全局的log
    \end{itemize}
    
    \subsubsection{辅助结构体}
    存储为了添加所用的所有信息
    \begin{itemize}
    \item {LineData}
        用来存储切分好的Line的相关信息,包括车次,所有车站名,时间等
    \item {ButReturnData}
        用来存储切分好的买票退票信息,包括买票人的ID、name、买票车次、座位类型、张数等
    \item {StationData}
        站名,以及这个站属于哪个城市
    \item {CityData}
        城市名
    \item {TrainData}
        车次,日期,卖票状态,车票余量(使用vector套vector)
    \end{itemize}

    \subsubsection{内部成员函数}
    \begin{itemize}
    \item query
        \begin{itemize}
        \item (bool) add_station(const StationData&)
        用一个station所需的所有信息,新加一个station
        \item (bool) add_line(const LineData&)
        用一个line所需的所有信息,新加一个line
        \item (bool) add_city(const CityData&)
        用一个city所需的所有信息,新加一个city
        \item (bool) add_train(const TrainData&)
        用一个train所需的所有信息,新加一个train
        \end{itemize}

    \subsubsection{API}
    \item void add_log(const QString &log)
    添加一条操作记录
    \item (vector<query_ticket_ans>) query_city_city(const query_ticket_cc_data & data)
    按照从城市到城市查询
    \item (vector<query_ticket_ans>) query_station_station(const query_ticket_ss_data & data)
    按照从车站到车站查询
    \item (vector<query_my_order_ans>) query_my_order(const query_my_order_data & data)
    查看我的订单
    
    \item (bool) add_line(const QString & data);
    添加一条线路,调用parser进行切分
    \item (login_user_ans) login_user(const login_user_data & data);
    用户登录
    \item (login_admin_ans) login_admin(const login_admin_data & data);
    管理员登录
    \item (return_tickets_ans) return_tickets(const return_tickets_data & data);
    退票(用户)
    \item (buy_tickets_ans) buy_tickets(const buy_tickets_data & data);
    买票(用户)

    \item (delete_line_ans) delete_line(const delete_line_data & data);
    删除线路(管理员)
    \item (register_user_ans) register_user(const register_user_data & data);
    注册用户
    \item (register_admin_ans) register_admin(const register_admin_data & data);
    注册管理员

    \subsubsection{辅助函数}
    \begin{itemize}
    \item {load_ascii()}
    从文本文档中读取初始信息
    \item {load_binary()}
    从二进制文件中读档
    \item {save_binary()}
    向二进制文件中存档
    \item {LineData line_transform(QString str)}
    将一个用逗号和换行符分割的QString转换成LineData类型的对象
    \item {BuyReturnData operation_transform(QString str)}
    将一行买票操作的信息字符串转换成一个BuyReturnData类型的对象
    \end{itemize}